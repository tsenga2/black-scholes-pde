\documentclass{beamer}
\usetheme{Madrid}
\usecolortheme{whale}
\usepackage{amsmath}
\usepackage{listings}
\usepackage{algorithm2e}
\usepackage{tikz}

\usepackage{algorithm}
\usepackage{algpseudocode}

% Configure listings for MATLAB code
\lstset{
    language=Matlab,
    basicstyle=\footnotesize\ttfamily,
    breaklines=true,
    captionpos=b,
    commentstyle=\color{green!60!black},
    keywordstyle=\color{blue},
    stringstyle=\color{purple},
    numbers=left,
    numberstyle=\tiny\color{gray},
    stepnumber=1,
    numbersep=5pt,
    backgroundcolor=\color{white},
    showspaces=false,
    showstringspaces=false,
    showtabs=false,
    tabsize=2,
    breakatwhitespace=false,
    escapeinside={(*@}{@*)}
}

\title{Implicit Finite Difference Method: \\ Black-Scholes Option Pricing}
\author{Tatsuro Senga}
\date{}

\begin{document}

\begin{frame}
    \titlepage
\end{frame}

\begin{frame}{Outline}
    \tableofcontents
\end{frame}

\section{Black-Scholes PDE}

\begin{frame}{The Black-Scholes Equation}
    \begin{align*}
        \frac{\partial V}{\partial t} + \frac{1}{2}\sigma^2S^2\frac{\partial^2 V}{\partial S^2} + (r-d)S\frac{\partial V}{\partial S} - rV = 0
    \end{align*}
    where:
    \begin{itemize}
        \item $V(S,t)$ is the option price
        \item $S$ is the stock price
        \item $t$ is time
        \item $\sigma$ is volatility
        \item $r$ is risk-free rate
        \item $d$ is dividend yield
    \end{itemize}
\end{frame}

\begin{frame}{Boundary Conditions}
    For a call option:
    \begin{align*}
        V(S,T) &= \max(S-K,0) \quad \text{(Terminal condition)} \\
        V(0,t) &= 0 \\
        V(S_{\text{max}},t) &= S_{\text{max}}-K
    \end{align*}
    
    For a put option:
    \begin{align*}
        V(S,T) &= \max(K-S,0) \quad \text{(Terminal condition)} \\
        V(0,t) &= K \\
        V(S_{\text{max}},t) &= 0
    \end{align*}
\end{frame}

\section{Discretization}
\begin{frame}{Grid Setup}
    \begin{itemize}
        \item Time discretization: $t_i = i\Delta t$, $i = 0,\ldots,N$
        \item Stock price discretization: $S_j = S_{\text{min}} + j\Delta S$, $j = 0,\ldots,M$
        \item $\Delta t = T/N$
        \item $\Delta S = (S_{\text{max}}-S_{\text{min}})/M$
    \end{itemize}
    \begin{center}
        \begin{tikzpicture}[scale=0.6]
            \draw[->] (0,0) -- (6,0) node[right] {$S$};
            \draw[->] (0,0) -- (0,4) node[above] {$t$};
            \draw[step=1cm] (0,0) grid (5,3);
            \node[below] at (2.5,0) {Stock price points};
            \node[left] at (0,1.5) {Time steps};
        \end{tikzpicture}
    \end{center}
\end{frame}

\begin{frame}{Finite Difference Approximations}
    \begin{align*}
        \frac{\partial V}{\partial t} + \frac{1}{2}\sigma^2S^2\frac{\partial^2 V}{\partial S^2} + (r-d)S\frac{\partial V}{\partial S} - rV = 0
    \end{align*}
    
    We approximate this using:
    \begin{align*}
        \frac{\partial V}{\partial t} &\approx \frac{V_j^{i+1} - V_j^i}{\Delta t} \\
        \frac{\partial V}{\partial S} &\approx \frac{V_{j+1}^i - V_{j-1}^i}{2\Delta S} \\
        \frac{\partial^2 V}{\partial S^2} &\approx \frac{V_{j+1}^i - 2V_j^i + V_{j-1}^i}{(\Delta S)^2}
    \end{align*}
    where $V_j^i$ represents the option value at $(S_j, t_i)$
\end{frame}

\begin{frame}{Implicit Scheme Derivation (1)}
    Substituting finite differences into Black-Scholes PDE:
    \begin{align*}
        &\frac{V_j^{i+1} - V_j^i}{\Delta t} + \frac{1}{2}\sigma^2S_j^2\frac{V_{j+1}^i - 2V_j^i + V_{j-1}^i}{(\Delta S)^2} \\
        &+ (r-d)S_j\frac{V_{j+1}^i - V_{j-1}^i}{2\Delta S} - rV_j^i = 0
    \end{align*}
    where $S_j = S_{\text{min}} + j\Delta S$
\end{frame}

\begin{frame}{Implicit Scheme Derivation (2)}
    Rearranging terms:
    \begin{align*}
        &V_j^i - V_j^{i+1} + \frac{1}{2}\sigma^2j^2\Delta t(V_{j+1}^i - 2V_j^i + V_{j-1}^i) \\
        &+ (r-d)j\frac{\Delta t}{2}(V_{j+1}^i - V_{j-1}^i) - r\Delta tV_j^i = 0
    \end{align*}
    Define coefficients:
    \begin{align*}
        a_j &= \frac{1}{2}(r-d)j\Delta t - \frac{1}{2}\sigma^2j^2\Delta t \\
        b_j &= 1 + \sigma^2j^2\Delta t + r\Delta t \\
        c_j &= -\frac{1}{2}(r-d)j\Delta t - \frac{1}{2}\sigma^2j^2\Delta t
    \end{align*}
\end{frame}

\begin{frame}{Implicit Scheme Derivation (3)}
    Final form:
    \[a_jV_{j-1}^i + b_jV_j^i + c_jV_{j+1}^i = V_j^{i+1}\]
    For all interior points $(j = 1,\ldots,M-1)$:
    \begin{itemize}
        \item Tridiagonal system at each time step
        \item Solved backwards in time ($i = N-1,\ldots,0$)
        \item Boundary conditions used at $j = 0$ and $j = M$
    \end{itemize}
\end{frame}


\begin{frame}{Implicit Scheme}
    \begin{itemize}
        \item After substituting finite differences:
        \begin{align*}
            &a_j V_{i}^{j-1} + b_j V_{i}^j + c_j V_{i}^{j+1} = V_{i+1}^j
        \end{align*}
        where:
        \begin{align*}
            a_j &= \frac{1}{2}(r-d)j\Delta t - \frac{1}{2}\sigma^2j^2\Delta t \\
            b_j &= 1 + \sigma^2j^2\Delta t + r\Delta t \\
            c_j &= -\frac{1}{2}(r-d)j\Delta t - \frac{1}{2}\sigma^2j^2\Delta t
        \end{align*}
    \end{itemize}
\end{frame}

\begin{frame}{Matrix Form}
    \begin{itemize}
        \item The system can be written as:
        \[A\vec{V}_i = \vec{V}_{i+1}\]
        \item Where $A$ is tridiagonal:
        \[A = \begin{pmatrix}
            b_1 & c_1 & 0 & \cdots & 0 \\
            a_2 & b_2 & c_2 & \cdots & 0 \\
            0 & a_3 & b_3 & \cdots & 0 \\
            \vdots & \vdots & \vdots & \ddots & \vdots \\
            0 & 0 & 0 & \cdots & b_M
        \end{pmatrix}\]
    \end{itemize}
\end{frame}

\begin{frame}{Solution Algorithm}
    \begin{algorithm}[H]
    \caption{Implicit Finite Difference Method}
    \begin{algorithmic}[1]
        \State Initialize grid and boundary conditions
        \State Set terminal conditions $V_N^j$ for all $j$
        \For{$i = N-1$ to $0$}
            \State Form tridiagonal matrix $A$
            \State Solve $A\vec{V}_i = \vec{V}_{i+1}$
            \State Apply free boundary condition
        \EndFor
    \end{algorithmic}
    \end{algorithm}
\end{frame}

\begin{frame}[fragile]{Tridiagonal Coefficients}
    The coefficients are:
    \begin{align*}
        a_j &= \frac{\Delta t}{2}[(r-d)j - \sigma^2j^2] \\
        b_j &= 1 + \sigma^2j^2\Delta t + r\Delta t \\
        c_j &= -\frac{\Delta t}{2}[(r-d)j + \sigma^2j^2]
    \end{align*}
    
    In MATLAB code:
    \begin{lstlisting}[language=Matlab]
a = @(j) 0.5*(r-d)*j*dt - 0.5*volatility^2*j^2*dt;
b = @(j) 1 + volatility^2*j^2*dt + r*dt;
c = @(j) -0.5*(r-d)*j*dt - 0.5*volatility^2*j^2*dt;
    \end{lstlisting}
\end{frame}

\begin{frame}{Matrix System}
    At each time step, we solve:
    \[
    \begin{pmatrix}
        b_1 & c_1 & 0 & \cdots & 0 \\
        a_2 & b_2 & c_2 & \cdots & 0 \\
        0 & a_3 & b_3 & \cdots & 0 \\
        \vdots & \vdots & \vdots & \ddots & \vdots \\
        0 & 0 & 0 & \cdots & b_M
    \end{pmatrix}
    \begin{pmatrix}
        V_1^i \\
        V_2^i \\
        V_3^i \\
        \vdots \\
        V_M^i
    \end{pmatrix}
    =
    \begin{pmatrix}
        V_1^{i+1} - a_1V_0^i \\
        V_2^{i+1} \\
        V_3^{i+1} \\
        \vdots \\
        V_M^{i+1} - c_MV_{M+1}^i
    \end{pmatrix}
    \]
\end{frame}

\section{Implementation}

\begin{frame}[fragile]{MATLAB Implementation}
    Key steps in the code:
    \begin{lstlisting}
    % Create tridiagonal matrix
    A = diag(a(2:M-1),-1) + diag(b(2:M)) 
        + diag(c(1:M-2),1);
    
    % Set up right-hand side
    v = surf(i+1,2:M)';
    v(1) = v(1) - a(1)*surf(i,1);
    v(end) = v(end) - c(M+1)*surf(i,M+1);
    
    % Solve system
    surf(i,2:M) = A\v;
    \end{lstlisting}
\end{frame}

\begin{frame}{Call Option Price}
            \includegraphics[width=\textwidth]{OptionPrice.png}
\end{frame}

\begin{frame}{Call Option Price Profile and Time Value}
    \begin{columns}
        \begin{column}{0.5\textwidth}
            \vspace{-0.5cm}
            \includegraphics[width=\textwidth]{OptionPriceSlice.png}
            \vspace{0.3cm}
            
            \textbf{Option Price Components:}
            \begin{itemize}
                \item \textcolor{blue}{Blue line}: Total option value
                    \begin{itemize}
                        \item = Intrinsic value + Time value
                    \end{itemize}
                \item \textcolor{red}{Red dashed}: Intrinsic value
                    \begin{itemize}
                        \item = max$(S-K, 0)$
                    \end{itemize}
                \item Time value = Blue $-$ Red
            \end{itemize}
        \end{column}
        
        \begin{column}{0.5\textwidth}
            \textbf{Time Value Analysis:}
            \begin{itemize}
                \item Out-of-money ($S < K$):
                    \begin{itemize}
                        \item Pure time value
                        \item Decays with distance from $K$
                    \end{itemize}
                \item At-the-money ($S = K = 50$):
                    \begin{itemize}
                        \item Maximum time value
                        \item ≈ $0.4K\sigma\sqrt{T}$ (rule of thumb)
                    \end{itemize}
                \item In-the-money ($S > K$):
                    \begin{itemize}
                        \item Time value decreases
                        \item Approaches intrinsic value
                    \end{itemize}
            \end{itemize}
            
            \textbf{Parameters:} $T = 1$ year, $\sigma = 40\%$, $r = 2\%$, $K = 50$
        \end{column}
    \end{columns}
\end{frame}

\section{Numerical Considerations}

\begin{frame}{Grid Selection}
    Important factors:
    \begin{itemize}
        \item Choice of $S_{\text{max}}$: typically 3-4 times strike price
        \item Number of time steps ($N$)
        \item Number of stock price steps ($M$)
        \item Trade-off between accuracy and computational cost
    \end{itemize}
    
    Recommendation:
    \begin{itemize}
        \item Start with $N = M = 100$
        \item Increase if more accuracy needed
        \item Check convergence by doubling grid points
    \end{itemize}
\end{frame}

\section{Detailed Implementation}


\begin{frame}[fragile]{Function Definition and Grid Setup}
    \begin{lstlisting}[language=Matlab]
function [t_vals,S_vals,surf] = black_scholes_naive_implicit(...
    N,M,Smin,Smax,T,K,volatility,r,d,is_call)
% Initialize the solution surface
surf = zeros(1+N,1+M);  % Size: (time steps + 1) × (price steps + 1)
    \end{lstlisting}
    
    Parameters explanation:
    \begin{itemize}
        \item \texttt{N}: Number of time steps
        \item \texttt{M}: Number of stock price steps
        \item \texttt{Smin, Smax}: Stock price range
        \item \texttt{T}: Time to maturity
        \item \texttt{K}: Strike price
        \item \texttt{volatility, r, d}: Market parameters
        \item \texttt{is\_call}: Option type flag
    \end{itemize}
\end{frame}

\begin{frame}[fragile]{Grid Construction}
    \begin{lstlisting}[language=Matlab]
% Determine step sizes
dt = T/N;      % Time step size
dS = (Smax-Smin)/M;  % Stock price step size

% Create grid points
t_vals = 0:dt:T;     % Time points array
S_vals = Smin:dS:Smax;  % Stock price array
    \end{lstlisting}
    
    Grid Details:
    \begin{itemize}
        \item Time grid: $[0, T]$ divided into $N$ steps
        \item Stock price grid: $[S_{min}, S_{max}]$ divided into $M$ steps
        \item \texttt{t\_vals}: Vector of length $N+1$
        \item \texttt{S\_vals}: Vector of length $M+1$
    \end{itemize}
\end{frame}

\begin{frame}[fragile]{Boundary Conditions}
    \begin{lstlisting}[language=Matlab]
% Set boundary conditions
if is_call
    surf(:,1) = 0;            % At S = Smin
    surf(:,end) = Smax-K;     % At S = Smax
    surf(end,:) = payoff(S_vals,K,is_call);  % At T
else
    surf(:,1) = K;            % At S = Smin
    surf(:,end) = 0;          % At S = Smax
    surf(end,:) = payoff(S_vals,K,is_call);  % At T
end
    \end{lstlisting}
    
    Explanation:
    \begin{itemize}
        \item \texttt{surf(:,1)}: Left boundary ($S = S_{min}$)
        \item \texttt{surf(:,end)}: Right boundary ($S = S_{max}$)
        \item \texttt{surf(end,:)}: Terminal condition at $T$
    \end{itemize}
\end{frame}

\begin{frame}[fragile]{Tridiagonal System Setup}
    \begin{lstlisting}[language=Matlab]
% Define tridiagonal matrix coefficients
a = @(j) 0.5*(r-d)*j*dt - 0.5*volatility^2*j^2*dt;  % Lower
b = @(j) 1 + volatility^2*j^2*dt + r*dt;            % Main
c = @(j) -0.5*(r-d)*j*dt - 0.5*volatility^2*j^2*dt; % Upper
    \end{lstlisting}
    
    These coefficients come from discretizing the PDE:
    \begin{itemize}
        \item \texttt{a(j)}: Coefficient of $V_{j-1}^i$ (lower diagonal)
        \item \texttt{b(j)}: Coefficient of $V_j^i$ (main diagonal)
        \item \texttt{c(j)}: Coefficient of $V_{j+1}^i$ (upper diagonal)
    \end{itemize}
\end{frame}

\begin{frame}[fragile]{Time-Stepping Loop}
    \begin{lstlisting}[language=Matlab]
for i = N:-1:1  % Backward in time
    % Construct tridiagonal matrix
    A = diag(a(2:M-1),-1) + diag(b(2:M)) + ...
        diag(c(1:M-2),1);
    
    % Set up right-hand side vector
    v = surf(i+1,2:M)';
    v(1) = v(1) - a(1)*surf(i,1);        % Left boundary
    v(end) = v(end) - c(M+1)*surf(i,M+1);  % Right boundary
    
    % Solve linear system
    surf(i,2:M) = A\v;
    \end{lstlisting}
    
    At each time step:
    \begin{itemize}
        \item Build tridiagonal matrix \texttt{A}
        \item Construct RHS vector \texttt{v} with boundary adjustments
        \item Solve system $AV^i = v$ for interior points
    \end{itemize}
\end{frame}

%\begin{frame}[fragile]{American Option Early Exercise}
%    \begin{lstlisting}[language=Matlab]
% Enforce free boundary condition (early exercise)
%surf(i,2:M) = max(surf(i,2:M), ...
%    payoff(S_vals(2:M),K,is_call));
%    \end{lstlisting}
    
%    Early Exercise Feature:
%    \begin{itemize}
%        \item Compare continuation value with immediate exercise
%        \item Take maximum of the two values
%        \item This makes the option American-style
%        \item Without this line, we get European options
%    \end{itemize}
    
%    Note: This is called a free boundary condition because the early exercise boundary is determined as part of the solution.
%\end{frame}

\begin{frame}{Matrix Structure Visualization}
    Example for $M=5$:
    \[
    \begin{pmatrix}
        b_2 & c_2 & 0 & 0 \\
        a_3 & b_3 & c_3 & 0 \\
        0 & a_4 & b_4 & c_4 \\
        0 & 0 & a_5 & b_5
    \end{pmatrix}
    \begin{pmatrix}
        V_2^i \\
        V_3^i \\
        V_4^i \\
        V_5^i
    \end{pmatrix}
    =
    \begin{pmatrix}
        V_2^{i+1} - a_2V_1^i \\
        V_3^{i+1} \\
        V_4^{i+1} \\
        V_5^{i+1} - c_5V_6^i
    \end{pmatrix}
    \]
    
    \begin{itemize}
        \item Matrix is tridiagonal
        \item System solved efficiently using backslash operator
        \item First and last rows modified by boundary conditions
    \end{itemize}
\end{frame}

\end{document}